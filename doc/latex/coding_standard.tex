% Copyright (C) 2009, INRIA
% Author(s): Vivien Mallet
%
% You can redistribute this document and/or modify it under the terms of the
% GNU General Public License as published by the Free Software Foundation;
% either version 2 of the License, or (at your option) any later version.
%
% This document is distributed in the hope that it will be useful, but WITHOUT
% ANY WARRANTY; without even the implied warranty of MERCHANTABILITY or
% FITNESS FOR A PARTICULAR PURPOSE. See the GNU General Public License for
% more details.

\newcounter{points}

\newcommand{\cscode}[1]{\texttt{#1}}
\newcommand{\csrule}[1]{\textsc{#1}}
\newcommand{\cscomment}[1]{\newline #1}
\newcommand{\cscommentf}[1]{#1}
\newcommand{\csjustification}[1]{\newline {\it Justification} --- #1}
\newcommand{\csnjustification}[1]{{\it Justification} --- #1}
\newcommand{\csjustificationf}[1]{#1}

\newenvironment{cenumerate}
{
  \begin{enumerate}\setcounter{enumi}{\value{points}}%
  }
  {
    \setcounter{points}{\value{enumi}}\end{enumerate}
}

%%%%%%%%%%%%%%%%%%
%% INTRODUCTION %%
%%%%%%%%%%%%%%%%%%


\section*{Introduction}
\label{part:introduction}

Many coding standards may be found on-line: see, for instance, the list
provided at \url{http://www.chris-lott.org/resources/cstyle/}. In particular,
the \textit{C++ Coding Standard} by Todd Hoff, available at
\url{http://www.possibility.com/Cpp/CppCodingStandard.html} provided useful
guidelines for the present standard.

The general conventions of part~\ref{part:general} shall be applied to all
computer codes, in any language. C++ conventions follow in
part~\ref{part:cpp}.


%%%%%%%%%%%%%%%%%%%
%% GENERAL RULES %%
%%%%%%%%%%%%%%%%%%%


\section{General Conventions}
\label{part:general}

The following conventions must be applied in all languages.

\subsection{Development}

\begin{enumerate}
\item \csrule{Codes must be written in American English. This includes variable
    names, file names, comments,~\ldots{}} \csjustification{A code should be
    open to anyone.  Even if a code is not intended to be shared initially,
    this may change.  Moreover parts of it may be reused in another project.}
\item \csrule{Optimize only the parts of the code that lead to significant
    overheads.} \cscomment{In addition, remember that the compilers partially
    optimize your code.} \csjustification{An optimized code is usually less
    readable than its straightforward version.}
\item \csrule{A code must compile without warnings.}  \cscomment{At development
    stage, compile your code with the most restrictive compilation options and
    with all warning messages enabled.} \csjustification{It increases
    portability, and it may help avoiding mistakes.}
\item \csrule{Global variables must be avoided.} \csjustification{Units of a code
    should be independent from their environment so that they may be reused
    and so that the overall code may be safer.}
  \setcounter{points}{\value{enumi}}
\end{enumerate}


\subsection{Names}

\begin{cenumerate}
\item \csrule{Use explicit and meaningful names.} \cscomment{You should first
    consider that the length does not matter. Actually it does, but many
    programmers tend to use too short and therefore uninformative
    names. Usually, the wider scope, the longer name.}  \csjustification{New
    programmers should be able to read the code without learning the meaning
    of the variables and the functions in use. Explicit names also save
    unnecessary comments.  Example (in C++):}
\begin{verbatim}
if (Species.GetName() == target_species)
    output_concentration = Species.Concentration();
\end{verbatim}
  \csjustificationf{is better than:}
\begin{verbatim}
// Checks whether current species is the target species.
if (sp.GetName() == species)
    // Retrieves the output concentration.
    c = sp.Concentration();
\end{verbatim}
\item \csrule{Avoid contractions. If you really need a contraction, only use it
    as a suffix.} \cscomment{Accepted contractions are: tmp (temporary), in
    (input), out (output), obs (observation). Example: \cscode{value\_in}.}
  \csjustification{Contractions can easily collide (including with a full word)
    or have meanings depending on the context. Besides, not all developers may
    understand your contractions, depending on their habits and maybe their
    native language. For instance, French developers often contract
    \cscode{number} into \cscode{nb} whereas a native English speaker would use
    \cscode{num}.}
\item \csrule{Unless a more explicit name is found, preferably use {\rm
      \cscode{i}} for an index. As for dimensions, use {\rm \cscode{h}} for time
    index, {\rm \cscode{i}} along x, {\rm \cscode{j}} along y and {\rm \cscode{k}}
    along z.} \csjustification{This is a common practice.}
\item \csrule{Avoid plural forms.} \cscomment{Even the name of a vector should not
    be in plural form. For instance, a vector of observations can be called
    \cscode{observation}, and the i-th observation is \cscode{observation(i)}
    which is perfectly clear. If the plural form seems to be required,
    consider appending \cscode{\_list}: e.g., \cscode{location =
      location\_list(i)}, even if \cscode{location\_list} is not an instance of
    the STL class \cscode{list}.} \csjustification{The use of plural forms makes
    it difficult to guess or remember the names of variables and methods. It
    is quickly unclear whether the plural form was used or not for such and
    such methods. The same is true with a complex object that stores many data
    sets: should it be named with plural form because it contains a list, or
    should the singular be preferred because it appears as a single
    block/object? The easiest way to avoid the confusion is to avoid plural
    forms.}
\item \csrule{A fixed number of \cscode{xxx} should be named \cscode{Nxxx}.}
  \cscomment{Example: \cscode{Nstep} for the number of steps (if it is fixed),
    \cscode{Narray} for a number of arrays.}
\item \csrule{The Boolean methods should have a prefix like \cscode{Is} or
    \cscode{Has}.} \cscomment{Examples: \cscode{IsReady}, \cscode{HasObservation},
    \cscode{IsEmpty}.} \csjustification{It makes it clear that the method returns
    a Boolean. The question that the method answers is very clear too.}
\end{cenumerate}


\subsection{Formatting}

\subsubsection{Spaces and Parens}
\label{sec:spaces-parens}

\begin{cenumerate}
\item \csrule{Put one blank space before and one blank space after the
    operators: \cscode{+}, \cscode{-}, \cscode{/}, \cscode{*} (multiplication),
    \cscode{=}, \cscode{+=}, \cscode{-=}, \cscode{*=}, \cscode{/=}, \cscode{|}, \cscode{\&},
    \cscode{||}, \cscode{\&\&}, \cscode{<}, \cscode{<=}, \cscode{>}, \cscode{>=},
    \cscode{==}, \cscode{!=}, \cscode{<{}<}.}  \cscomment{You might break this rule in
    inner parens (at a deep level, e.g., some array index like \cscode{i+1} in a
    complex formula).} \csjustification{It makes the code much more readable. It
    is also a very common practice.}
\item \csrule{Put one blank space after each comma.} \csjustification{It makes the
    code more readable. It is a very common practice.}
\item \csrule{Do not add trailing spaces at the end of code lines.} \cscomment{A
    script can remove the trailing spaces for you---such a script should be
    run before any commit to the repository of your revision control
    system. Emacs users may have their editor removing the trailing spaces
    whenever they save a file. They can also have Emacs show them the trailing
    whitespaces; e.g., in Python mode:}
\begin{verbatim}
(setq whitespace-style '(trailing))
(add-hook 'python-mode-hook 'whitespace-mode)
\end{verbatim}
  \csnjustification{Browsing the code, moving blocks, copies,~\ldots{} are
    slowed down because of trailing spaces. In addition, the differences
    between two revisions of a file should not include such noise.}
\item \csrule{No space between the function name and its arguments list.}
  \cscomment{Write}
\begin{verbatim}
  func(a, b)
\end{verbatim}
  \cscommentf{instead of}
\begin{verbatim}
  func (a, b)
\end{verbatim}
\item \csrule{No space after an opening paren or a closing paren.}
  \cscomment{Write}
\begin{verbatim}
  func(a, b)
\end{verbatim}
  \cscommentf{instead of}
\begin{verbatim}
  func( a, b )
\end{verbatim}
\item \csrule{Put a space between a language keyword and the following paren.}
  \cscomment{Write}
\begin{verbatim}
  while (error > epsilon)
\end{verbatim}
  \cscommentf{instead of}
\begin{verbatim}
  while(error > epsilon)
\end{verbatim}
  \csnjustification{Keywords and functions should be distinguishable.}
\item \csrule{Do not put unnecessary parens in logical expressions, except to
    clarify the order of evaluation.}  \cscomment{For instance (in C++), write}
\begin{verbatim}
  if (i != 0 && j > 5)
\end{verbatim}
\cscommentf{instead of}
\begin{verbatim}
  if ((i != 0) && (j > 5))
\end{verbatim}
\csnjustification{Unncessary parens slow down the reading.}
\end{cenumerate}

\subsubsection{Comments}

\begin{cenumerate}
\item \csrule{A comment line is placed before the lines or the block it
    comments.}  \cscomment{Observe where each comment line is placed:}
\begin{verbatim}
// Checks the availability of observations. Note that the call implicitly
// loads the observations at current date.
if (observation_manager.HasObservation())
// Assimilates the available observations.
{
    Analyze();
    if (positive_state)
        // Enforces the positivity of the state vector.
        for (int i = 0; i < Nstate; i++)
            state(i) = max(0., state(i));
}
\end{verbatim}
  \csnjustification{Since an explanation may address several lines, the scope of
    a comment placed after is unclear.}
\item \csrule{Like a sentence, a comment starts with a capital and ends with a
    dot (even a one-word comment).}  \csjustification{All comment lines should
    be consistent. This rule is clear and easy to follow.}
\item \csrule{Use simple present to explain what a line does or what a sequence
    of lines does.} \cscomment{See the example above.} \csjustification{A comment
    introduces to what a line {\it does}. So \cscode{// Extracts data.}
    implicitly means \cscode{// This line extracts data.}}
\item \csrule{When referring to a variable, surround the variable name with
    simple quotes.} \cscomment{Example: \cscode{// Updates 'state' so that it
      should be consistent with 'full\_state'.}}  \csjustification{Reading
    comments, with variable names included, can be really difficult because
    the variable names are often words: one cannot identify at first sight
    that the variable name is a special element. Think of a comment line like
    \cscode{// Makes a consistent with the location.}  instead of \cscode{// Makes
      'a' consistent with the location.}}
\end{cenumerate}

\subsubsection{Function and Method Definition}

\begin{cenumerate}
\item \csrule{Arguments are sorted from input variables to output variables.
    Dimensions are provided first.} \cscomment{The only exception is for
    optional arguments if they are necessarily the last arguments (as in C++
    and Python). Try to sort all arguments so that the order makes sense. For
    instance, if the input variables are a number of points along $x$, the
    abscissae and the values of a function $f$ at these abscissae, then
    provide them in that order. Indeed, one needs first the number of points
    $n$, then the positions $x_i$ of the points ($i\in\llbracket{}0,
    n-1\rrbracket$) and finally the associated values $f(x_i)$.}
  \csjustification{A code line is read from left to right, and obviously an
    input comes before an output.}
\end{cenumerate}


\subsection{Language Features}

\subsubsection{Standard}

\begin{cenumerate}
\item \csrule{Build a fully standard-compliant code. If a compiler does not
    understand it, discard it or maybe maintain specific code for it.}
  \csjustification{This ensures portability and makes the code perennial.}
\end{cenumerate}


%%%%%%%%%
%% C++ %%
%%%%%%%%%


\section{C++}
\label{part:cpp}


\subsection{Development}

\begin{cenumerate}
\item \csrule{Compilation with GNU G++ and with options {\rm \cscode{-Wall -ansi
        -pedantic}} should not issue any warning.} \csjustification{It increases
    portability (through compliance with the C++ standard), and it may help
    avoiding mistakes.}
\end{cenumerate}


\subsection{Names}

\subsubsection{Common Conventions}

\begin{cenumerate}
\item \csrule{Do not add a prefix to a set of objects to avoid conflicts.}
  \cscomment{Use name spaces instead.}
\end{cenumerate}

\subsubsection{Classes and Methods}

\begin{cenumerate}
\item \csrule{A name is one word or a concatenation of words. The first letter
    of each word (including the first one) is uppercase.}  \cscomment{Examples:}
\begin{verbatim}
class FormatBinary;
double Data::GetMax() const;
void Data::Print() const;
void LoadPreviousAnalysis(...);
\end{verbatim}
  \csnjustification{Two other conventions are widely used: \cscode{formatBinary}
    and \cscode{format\_binary}. With the latter convention, classes and methods
    are not easily identified in a code. The former convention is a bit
    inconsistent: one-word methods are not as emphasized as two-word methods
    are.}
\item \csrule{The accessors should be named with {\rm \cscode{Get}} or {\rm
      \cscode{Set}} (prefixed).} \cscomment{Example: \cscode{GetDate},
    \cscode{SetPosition}}.
\end{cenumerate}

\subsubsection{Functions}

\begin{cenumerate}
\item \csrule{Same rules as for the methods.} \cscomment{Meanwhile, the name of a
    small function (e.g., a single formula, or an extension of the C++
    libraries) may be lowercase with underscores to delimit the words.}
\item \csrule{Extern functions should be lowercase with underscores to delimit
    the words, and they should be prefixed by an underscore.}  \cscomment{The
    name of an extern function is defined by the compiler. If the compiler
    does comply with this convention, define a macro that follows this
    convention. Example:}
\begin{verbatim}
#define _linear_interpolation linear_interpolation_
\end{verbatim}
  \csnjustification{Fortran functions are usually named with one or two
    underscores at the end. The rule is consistent with the addition of an
    underscore, but as a prefix in order to avoid conflicts (with attributes,
    see below).}
\end{cenumerate}

\subsubsection{(Local) Variables}

This section also applies to method arguments and function arguments.

\begin{cenumerate}
\item \csrule{Use lower case and words delimited with underscores.}
  \csjustification{Many variables are naturally lowercase, like the indexes. In
    addition, one may declare an instance of a class, say \cscode{Data} or
    \cscode{ObservationManager}, with the same name as the class: \cscode{Data
      data} or \cscode{ObservationManager observation\_manager}.}
\end{cenumerate}

\subsubsection{Attributes}

\begin{cenumerate}
\item \csrule{Same rules as for variables, except that an underscore must be
    appended at the end of the name.} \cscomment{This rule might be broken in
    case the attribute is public or for consistency with a public attribute
    that has no appended underscore.} \csjustification{The underscore at the end
    enables to distinguish attributes from local variables within the
    methods. The scope is an important property of a variable. In addition,
    the method arguments may have the same names as the attributes, without
    the underscore. For example:}
\begin{verbatim}
void ExtendedStream::SetDelimiter(delimiter)
{
    delimiter_ = delimiter;
}
\end{verbatim}
\end{cenumerate}

\subsubsection{References, Pointers, Global Variables and Constant Variables}

\begin{cenumerate}
\item \csrule{No special notation is associated with references, pointers,
    global variables or constant variables.} \csjustification{Constant variables
    are declared as such (keyword \cscode{const}); no alteration of these
    variables can occur.  Global variables should be avoided. References are
    used in C++ to manipulate variables just like others: a notation to
    distinguish them would break this advantage.  Programmers sometimes prefix
    a 'p' for pointers, but the syntax is usually clear enough to show that a
    pointer is in use.}
\end{cenumerate}

\subsubsection{Name Spaces}

\begin{cenumerate}
\item \csrule{Name spaces are mainly used for libraries. A name space has the
    exact name of its library.} \cscomment{Example:}
\begin{verbatim}
namespace Verdandi;
\end{verbatim}
\end{cenumerate}

\subsubsection{Type Names (typedef)}

\begin{cenumerate}
\item \csrule{Use lower case and words delimited with underscores. Do not append
    an underscore even if the type name is defined in a class.}
  \csjustification{Type names are used as shortcuts for what may be seen as a
    low-level type, just like a \cscode{string} or a \cscode{double} in the
    \cscode{main()} function.}
\end{cenumerate}

\subsubsection{Macros}

\begin{cenumerate}
\item \csrule{Use upper case and words delimited by underscores.}
  \csjustification{Macros should be clearly identified because of their really
    specific nature. In addition, this is common practice.}
\item \csrule{In case a macro is related to a library, the first word of the
    macro must be the library name.} \cscomment{Example:}
\begin{verbatim}
  #define VERDANDI_DEBUG_LEVEL_4
\end{verbatim}
  \csnjustification{This avoids conflicts with macros from other libraries, and
    it better indicates what the macro is for.}
\end{cenumerate}


\subsection{Formatting}
\label{sec:formatting}

\subsubsection{Indentation and Braces}
\label{sec:indentation-braces}

\begin{cenumerate}
\item \csrule{Use the Allman standard for indentation.} \cscomment{This
    indentation style is called BSD under Emacs. It looks like this:}
\begin{verbatim}
    if (i == f(a, b))
        i += 5;  // No braces for a single line.
    else
    {   // Instead of "else {".
        while (i != 5)
        {
            j = 3 * f(4, b);
            i++;
        }
        i--;
    }
\end{verbatim}
  \cscommentf{Note the (compulsory) 4-space depth for the indentation. Emacs
    users can enforce this convention with the following code (placed in
    \cscode{.emacs}):}
\begin{verbatim}
(defun verdandi-c++-mode ()
  (interactive)
  (c-set-style "bsd")
  (setq c-basic-offset 4)
  (add-hook 'before-save-hook 'delete-trailing-whitespace t t))
(defun verdandi-c++-mode-hook ()
  (if (string-match "verdandi" buffer-file-name)
      (verdandi-c++-mode)))
(add-hook 'c++-mode-hook 'verdandi-c++-mode-hook)
\end{verbatim}
  \cscommentf{The hook \cscode{delete-trailing-whitespace} will remove trailing
    spaces when the file is saved, which is another formatting rule (see
    section~\ref{sec:spaces-parens}). The Verdandi C++ mode will be applied to
    any file that contains the word ``verdandi'' in its absolute path.}
  \csjustification{Consistent indentation is utterly required at least to browse
    the code and to grasp its structure. In addition, differences between two
    versions of a same code are easier to parse with a standard indentation.}
\item \csrule{Tabulations are forbidden. They should be replaced with
    spaces.}\cscomment{Emacs users can add the following line in their
    \cscode{.emacs}:}
\begin{verbatim}
(setq-default indent-tabs-mode nil)
\end{verbatim}
  \csnjustification{Tabulation-based indentation may be more convenient for code
    browsing than space-based indentation: moving in the code may be
    faster. Unfortunately, the length of a tabulation may vary from one
    environment to another (usual values are 4 or 8 spaces, maybe 2
    sometimes). This can cause misalignment. Take for example:}
\begin{verbatim}
< TAB >if (condition)
< TAB >< TAB >long_function_name(a, b, c,
< TAB >< TAB >< TAB >< TAB >....h, j, i);
\end{verbatim}
  \csjustificationf{where \cscode{< TAB >} is a tabulation and \cscode{.} is a
    whitespace. If the tabulation length is decreased, the code may look
    misaligned:}
\begin{verbatim}
<TAB>if (condition)
<TAB><TAB>long_function_name(a, b, c,
<TAB><TAB><TAB><TAB>....h, j, i);
\end{verbatim}
  \csjustificationf{where \cscode{h, j, i} is not properly placed. A clever
    solution is to use tabulations for block indentation and to use spaces for
    alignment:}
\begin{verbatim}
< TAB >if (condition)
< TAB >< TAB >long_function_name(a, b, c,
< TAB >< TAB >...................h, j, i);
\end{verbatim}
  \csjustificationf{In that configuration, the formatting will be fine whatever
    the tabulation length. Unfortunately, it seems that only Emacs and vi
    implement that formatting. In addition, the interaction between such a
    convention and the limit on the line length may be an issue. To conclude,
    the safest indentation strategy is to use spaces only, which any decent
    editor should handle:}
\begin{verbatim}
....if (condition)
........long_function_name(a, b, c,
...........................h, j, i);
\end{verbatim}
\item \csrule{The indentation depth is four spaces.} \csjustification{The
    indentation depth is usually four spaces (like in GNU coding standards) or
    eight spaces (like in Linux kernel). In scientific computing, several
    indentation levels are commonly required---for example for loops in
    three-dimensional space. A eight-space indentation depth is not practical
    in that context.}
\item \csrule{Do not put a useless semi-colon at the end of a block.}
  \cscomment{Only classes and structures declarations require a semi-colon after
    the closing brace.}
\end{cenumerate}

\subsubsection{Comments and Documentation}
\label{sec:comm-docum}

\begin{cenumerate}
\item \csrule{Use \cscode{//}, not \cscode{/*}, except for Doxygen comments.}
\item \csrule{Include Doxygen comments for every class, every attribute, every
    method and every function.}
\item \csrule{With respect to Doxygen comments of functions and methods:
    \begin{itemize}
    \item there must be a brief description, introduced by {\rm \cscode{//!}},
      or {\rm \cscode{$\backslash$*! $\backslash$brief}} if the description does
      not fit into a single line;
    \item there must be a description for every argument and for the returned
      value:
      \begin{itemize}
      \item only use {\rm \cscode{$\backslash$param[in]}}, {\rm
          \cscode{$\backslash$param[in,out]}} and {\rm
          \cscode{$\backslash${param[out]}}} to introduce the description for an
        argument;
      \item the description for an argument starts lowercase and ends with a
        dot;
      \item the description for a returned value starts with a capitalized
        word and ends with a dot.
      \end{itemize}
    \item a full description should also be added whenever necessary, just
      before the arguments description;
    \item any reference to an argument, say {\rm \cscode{x}}, should be
      introduced with the Doxygen command {\rm \cscode{$\backslash$a}}: {\rm
        \cscode{$\backslash$a x}}.
    \end{itemize}~}
    \cscomment{Template:}
\begin{verbatim}
//! Here comes the brief description.
/*! Here comes the long description, if needed.
  \param[in] x description of the first parameter. It may include several
  sentences on several lines.
  \param[in,out] value another parameter.
  \param[out] error true if and only if an error occurred.
*/
\end{verbatim}
  \cscommentf{Another template:}
\begin{verbatim}
/*! \brief Here goes a brief description that does not fit into a single
  line. */
/*! Here comes the long description, if any.
  \param[in] value description of the argument.
  \return The sign of \a value.
*/
\end{verbatim}
  \csnjustification{These rules ensure that the Doxygen documentation is
    complete and clean.}
\item \csrule{Doxygen comments for functions and methods are put before the
    definition in the source file, not before the declaration in the header
    file.} \csjustification{Headers should be as light as possible so that one
    may browse them quickly.}
\item \csrule{Doxygen comments for classes and attributes are put before their
    declarations, in the header file.} \cscomment{For attributes, the
    description is introduced by {\rm \cscode{//!}}, or {\rm
      \cscode{$\backslash$*! $\backslash$brief}} if it does not fit into a
    single line.} \csjustification{There is no other suitable place.}
\item \csrule{The code may be organized in sections and subsections that are
    introduced as follows.}  \cscomment{Section:}
\begin{verbatim}
  <two blank lines>
  ////////////////////////
  // READS INPUT FIELDS //
  ////////////////////////
  <two blank lines>
\end{verbatim}
  \cscommentf{Long section (especially in libraries, with many functions in the
    section---otherwise prefer the previous format):}
\begin{verbatim}
  <two blank lines>
  ///////////////////////////
  // MATRIX FACTORIZATIONS //
  <two blank lines>
  [ code of the long section ]
  <two blank lines>
  // MATRIX FACTORIZATIONS //
  ///////////////////////////
  <two blank lines>
\end{verbatim}
\cscommentf{Heavy subsection:}
\begin{verbatim}
  <two blank lines>
  /****************
   * Binary files *
   ****************/
  <two blank lines>
\end{verbatim}
\cscommentf{Light subsection (prefer this to the heavy subsection, except if the
  subsection is long---you may then nest light subsections inside the heavy
  subsection):}
\begin{verbatim}
  <one blank line>
  /*** Binary files ***/
  <one blank line>
\end{verbatim}
\item \csrule{Name spaces are inserted this way:}
\begin{verbatim}
<two blank lines>
namespace AtmoData
{
<two blank lines>
    [ code ]
<two blank lines>
} // namespace AtmoData.
<two blank lines>
\end{verbatim}
\end{cenumerate}

\subsubsection{Lines}
\label{sec:lines}

\begin{cenumerate}
\item \csrule{Only one statement should be put on a single line.}
  \cscomment{Write}
\begin{verbatim}
int j;
string line;
if (i == 5)
    cout << "Done." << endl;
\end{verbatim}
  \cscommentf{instead of}
\begin{verbatim}
int j; string line;
if (i == 5) cout << "Done." << endl;
\end{verbatim}
  \cscommentf{There should never be two semi-colons on the same line, except in
    \cscode{for} statements:}
\begin{verbatim}
for (i = 0; i < 10; i++)
    cout << i << endl;
\end{verbatim}
  \csnjustification{This usually makes the code clearer. It avoids misleading
    implementations like:}
\begin{verbatim}
if (i == 5)
    cout << "Done." << endl; i++;
\end{verbatim}
\item \csrule{Functions and methods definitions should be separated by two blank
    lines.}  \csjustification{There should be more space between two functions
    than between two blocks in a function.}
\item \csrule{A line should not contain strictly more than 78 characters.}
  \csjustification{This ensures that the code may be properly printed and that
    it could be displayed on a screen in text mode. Furthermore, one often
    needs to display two source files side by side. The differences between
    two files can also be displayed side by side.}
\item \csrule{If you need to break a line that introduces the definition of a
    method, never separate the two colons \cscode{::} from the name of the
    method.} \cscomment{Write}
\begin{verbatim}
ClassName
::MethodName(...long argument list...)
\end{verbatim}
  \cscommentf{instead}
\begin{verbatim}
ClassName::
MethodName(...long argument list...)
\end{verbatim}
  \csnjustification{One often searches for the definition of a given method in a
    source file. With this rule, one can search for \cscode{::MethodName} (using
    the search ability of one's text editor) to quickly find the definition. A
    search for \cscode{MethodName} may be inefficient because there may be many
    calls to the method, at different places.}
\end{cenumerate}

\subsubsection{Variable Definition}

\begin{cenumerate}
\item \csrule{Do not systematically declare all variables at the beginning of a
    program or a function. Declare small bunches of variables instead.}
  \cscomment{Old languages require that all variables are declared at the very
    beginning. This is the reason why this convention is still in use, even in
    modern programming languages. It is good idea to declare at the beginning
    a few variables that will be used at many places in the current block,
    like an index \cscode{i}. Otherwise, declare the variables more locally.}
  \csjustification{A variable declaration should be close to the lines where it
    is used so that the programmer can easily access to this declaration and
    so that the scope of the variable can be restricted.}
\item \csrule{Characters \cscode{*} and \cscode{\&} should be directly connected to
    the type, not the variable.} \cscomment{Write}
\begin{verbatim}
  int& i;
\end{verbatim}
\cscommentf{instead of}
\begin{verbatim}
  int &i;
\end{verbatim}
\csnjustification{In the previous example, the type of \cscode{i} is \cscode{int\&},
  and it should appear as such.}
\item \csrule{Constants are declared with a {\rm \cscode{const}} statement, not
    with {\rm \cscode{\#define}} or so.}
\end{cenumerate}

\subsubsection{Class Definition}

\begin{cenumerate}
\item \csrule{All attributes should be protected ({\rm \cscode{protected}}).}
  \cscomment{Public attributes are a bad idea because there are really
    unsafe. If there are so many attributes that maintaining accessors is
    difficult, this rule might be broken. Private attributes can be fine, but
    they cannot be accessed by the derived classes, which is rarely a useful
    feature.} \csjustification{Attributes should be accessed and modified only
    with methods, so as to allow miscellaneous checks, and so as to guaranty
    at any time the consistency of the object.}
\item \csrule{In a class definition, put, in this order:
    \begin{enumerate}
    \item the {\rm \cscode{typedef}} declarations;
    \item the attributes;
    \item the constructor(s);
    \item the destructor;
    \item the public methods;
    \item the protected methods;
    \item the private methods.
    \end{enumerate}In the source file, the definitions must follow the same
    order as the declarations in the header file.}
\item \csrule{Provide constant methods ({\rm \cscode{const}}) whenever possible.}
  \csjustification{It makes the code safer, and such methods are compulsory to
    manipulate \cscode{const} objects.}
\item \csrule{Declare a virtual destructor in case the class is derived and
    contains virtual methods.} \csjustification{If a pointer to an instance of a
    derived class is used, then } \cscode{delete} \csjustificationf{will only call
    the base destructor. In addition, some compilers will issue a warning.}
\end{cenumerate}

\subsubsection{Another Rule}

\begin{cenumerate}
\item \csrule{Floating-point numbers should always have a decimal point.}
  \cscomment{While manipulating floating-point numbers, write}
\begin{verbatim}
double x = 2.; // or 2.0
double y = 2. * x;
y = 1. + x / 3.;
\end{verbatim}
  \cscommentf{instead of}
\begin{verbatim}
double x = 2;
double y = 2 * x;
y = 1 + x / 3;
\end{verbatim}
  \csnjustification{It clearly shows what type of variable is manipulated. It
    can avoid mistakes like writing \cscode{2 / 3} (which is zero) instead of
    \cscode{2. / 3.} (which is of course not zero).}
\end{cenumerate}


\subsection{Files}
\label{sec:files}

\subsubsection{General Rule}

\begin{cenumerate}
\item \csrule{Definition shall never follow declaration. Put declarations in
    header files and definitions in source files. There must be a header file
    for any source file.}  \csjustification{First, a precompiled library can be
    built only if declarations and definitions have been split. Second, the
    contents of a library may be quickly browsed in its headers.}
\end{cenumerate}

\subsubsection{Names}

\begin{cenumerate}
\item \csrule{Extensions are {\rm \cscode{hpp}} or {\rm \cscode{hxx}} (for headers)
    and {\rm \cscode{cpp}} or {\rm \cscode{cxx}} (for sources). {\rm \cscode{*xx}}
    files should be used for libraries, exclusively.}  \csjustification{It is
    convenient to identify what is part of a software (to be compiled) and
    what is part of a library (to be included).}
\item \csrule{For libraries, a header file and a source file should be
    associated to each class. Those files have the same name as the class and
    they match its case.}
\item \csrule{The names of the directories and the names of the files to be
    compiled are lowercase with underscores to delimit the words.}
  \cscomment{Lower case is used for the files not part of the core library
    (\cscode{*.cpp}): examples, unit tests,~\ldots{}} \csjustification{Browsing
    the code from command line is easier with lowercase directory names, and
    it is consistent with Linux/Unix conventions. A file name of the core
    library should be the same as the class it implements (see above), but
    other files should be lowercase also for convenience and consistency with
    the environment.}
\end{cenumerate}

\subsubsection{Includes}

\begin{cenumerate}
\item \csrule{All library files must have include guards.}
\item \csrule{Include guards must be in the form \cscode{\{library name (upper
      case)\}\_FILE\_\{file name with its full path in the library (upper
      case; slashes and the dot are replaced with an underscore)\}}.}
  \cscomment{Example: \cscode{SELDON\_FILE\_SHARE\_VIRTUALALLOCATOR\_HXX} for the
    file ``VirtualAllocator.hxx'' in directory ``share'' of the library
    Seldon.}
\item \csrule{Include libraries of the C++ standard with \cscode{<} and \cscode{>},
    and other libraries with double quotes.} \cscomment{Example:}
\begin{verbatim}
#include <vector>
#include "Seldon.hxx"
\end{verbatim}
\end{cenumerate}


\subsection{About C++ Features}

\subsubsection{Exceptions}

\begin{cenumerate}
\item \csrule{Use exceptions to manage errors.} \csjustification{First, exception
    are in C++ precisely to manage errors. In old languages, one adds an
    integer to the arguments of all functions in order to track errors. This
    is difficult to maintain and it is dangerous because errors may be
    detected without further action. The other practice is to simply terminate
    the program (e.g., with \cscode{abort()}), but then the program cannot
    recover from the error.}
\item \csrule{Do not use exception specifications.} \csjustification{It is a
    nightmare to keep these exception specifications up-to-date because of
    exceptions that may be thrown by nested functions.}
\item \csrule{Check every memory allocation.} \cscomment{This is usually managed
    at a rather low level, in the libraries that provide the base structures.}
  \csjustification{If a memory allocation fails, a program should not keep
    running.}
\item \csrule{Allow the user to mute error checking.} \cscomment{To achieve this,
    enclose any test and its throw statement by} \cscode{\#ifdef \{library name
    (upper case)\}\_DEBUG\_\{small description\}} \cscommentf{and}
  \cscode{\#endif}.  \cscommentf{Example:}
  \cscode{VERDANDI\_DEBUG\_CHECK\_DIMENSION} \csjustification{Some tests may
    result in significant overheads. For instance, in the access to an element
    of a vector, checking the validity of the index requires a significant
    amount of time compared to the access itself. The test may be very
    helpful, but one should be able to deactivate it.}
\end{cenumerate}

\subsubsection{Templates}

\begin{cenumerate}
\item \csrule{For scientific computing, always consider template functions: most
    functions should have the numerical type as template parameter.}
  \cscomment{Do not fear templates! They are really helpful to build generic
    codes while maintaining high performance.}  \csjustification{In scientific
    computing, the type of the underlying data may change: \cscode{float},
    \cscode{double}, \cscode{complex<float>}, \cscode{complex<double>} or even a
    user-defined class. In addition, one never perfectly predicts what will be
    the eventual use of one's code; for instance, parts of it may be reused in
    another context, with other data structures.}
\end{cenumerate}

\subsubsection{C++ Standard}

\begin{cenumerate}
\item \csrule{Build a fully standard-compliant code. If a compiler does not
    understand it, discard it or write the code it needs within a {\rm
      \cscode{\#define}}/{\rm \cscode{\#endif}} block.} \csjustification{This
    ensures portability and makes the code perennial.}
\item \csrule{Do not use C features if they have been replaced by C++ features.}
  \cscomment{Even if C++ features do not always seem better at first sight,
    trust the designers of the C++ standard.} \csjustification{C++ features are
    better than their C equivalents, except that they might lead to overheads
    in a few cases.}
\end{cenumerate}


\subsection{Other Conventions}

\begin{cenumerate}
\item \csrule{In a loop, the stopping test should use an inclusive lower-bound
    or an exclusive upper-bound.}  \cscomment{Example:}
\begin{verbatim}
  for (unsigned int i = 0; i < 50; i++)
\end{verbatim}
\end{cenumerate}


%%%%%%%%%%%%
%% PYTHON %%
%%%%%%%%%%%%


\section{Python}
\label{part:python}

\begin{cenumerate}
\item \csrule{Tabulations are forbidden. They should be replaced with spaces.}
  \csjustification{In Python, the indentation is part of the language since it
    defines the blocks. If the indentation changes because of a varying
    tabulation length and a mixing of spaces and tabulations, the code
    changes. This is a risk nobody is willing to take.}
\item \csrule{The indentation depth is four spaces.} \csjustification{Same as in
    C++, see section~\ref{sec:indentation-braces}.}
\item \csrule{A line should not contain strictly more than 78 characters.}
  \csjustification{Same as in C++, see section~\ref{sec:lines}.}
\item \csrule{Use Doxygen, with similar to conventions to those for C++---section~\ref{sec:comm-docum}.}
\end{cenumerate}

This part of the document should be completed later. Note that the conventions
should be similar to C++. The ``Style Guide for Python Code'', by Guido van
Rossum and Barry Warsaw, \url{http://www.python.org/dev/peps/pep-0008/},
should serve as a sound background.


%%%%%%%%%%%%%%%%%%%%%
%% OTHER LANGUAGES %%
%%%%%%%%%%%%%%%%%%%%%


\section{Other Languages}

\begin{cenumerate}
\item \csrule{Only use a limited number of languages.} \cscomment{For instance,
    with C++ and Python, one can cover all needs: from low-level
    high-performance computing to high level programming (scripts, command
    line).}  \csjustification{It is better to fully understand a few languages
    than poorly using many languages. Moreover adding dependencies (to other
    languages and, as a consequence, to other compilers and libraries) may
    decrease the portability and may increase the installation difficulty.}
\item \csrule{Try to follow the rules associated with C++.} \cscomment{In case you
    {\it really} need to use another language, C++ conventions should cover
    most features of this additional language.}
\end{cenumerate}
